\documentclass[a4paper]{book}

\usepackage[colorlinks=true,linkcolor=blue,urlcolor=blue,citecolor=blue]{hyperref}
\usepackage[bahasa]{babel} 
\usepackage{indentfirst}
\usepackage[nottoc]{tocbibind} 

\usepackage[round]{natbib}

%\renewcommand{\chaptername}{BAB}
%\renewcommand{\bibname}{Daftar Pustaka}

%% https://github.com/ardhi175

\begin{document}
\tableofcontents

\chapter{Pendahuluan}

\section{Latar Belakang}

Blog : \url{https://mmfvlogs.blogspot.com}

Youtube : \url{https://youtube.com/c/MMFVlogs}

Ini pakai cite : Menurut \cite{Beiser2002-jh} dan \cite{Arfken2005-ce} dan \cite{Greiner2000-ks}

Ini pakai citet : Menurut \citet{Beiser2002-jh} dan \citet{Arfken2005-ce} dan \citet{Greiner2000-ks}

Ini pakai citep : Menurut \citep{Beiser2002-jh} dan \citep{Arfken2005-ce} dan \citep{Greiner2000-ks}

Ini pakai citealt : Menurut \citealt{Beiser2002-jh} dan \citealt{Arfken2005-ce} dan \citealt{Greiner2000-ks}

Ini pakai citealp : Menurut \citealp{Beiser2002-jh} dan \citealp{Arfken2005-ce} dan \citealp{Greiner2000-ks}

Ini pakai cite* : Menurut \cite*{Beiser2002-jh} dan \cite*{Arfken2005-ce} dan \cite*{Greiner2000-ks}

Ini pakai citet* : Menurut \citet*{Beiser2002-jh} dan \citet*{Arfken2005-ce} dan \citet*{Greiner2000-ks}

Ini pakai citep* : Menurut \citep*{Beiser2002-jh} dan \citep*{Arfken2005-ce} dan \citep*{Greiner2000-ks}

Ini pakai citealt* : Menurut \citealt*{Beiser2002-jh} dan \citealt{Arfken2005-ce} dan \citep*{Greiner2000-ks}

Ini pakai citealp* : Menurut \citealp*{Beiser2002-jh} dan \citealp*{Arfken2005-ce} dan \citealp*{Greiner2000-ks}

% style bisa diganti-ganti, lihat keterangan di bawah
\bibliographystyle{abbrvnat}
\bibliography{daftarpustaka.bib}

\end{document}


%% Prosedur menggenerate bibliography setiap setelah mengganti style :
%% -------------------------------------------------------------------
%% 1. Dari file utama, run Quick Build 1x
%% 2. Dari file utama, run BibTex 1x
%% 3. Dari file utama, run Quick Build 2x

%% Keterangan package yang dipakai :
%% ---------------------------------
%% \usepackage{indentfirst} : untuk memberikan indentasi pada setiap paragraf pertama setiap section
%% \usepackage[nottoc]{tocbibind} %untuk menambahkan bibliography pada daftar isi
%% \usepackage[]{natbib} : Memberikan tambahan mode sitasi : \citet,\citep,\citealt,\citealp, dll. Untuk lebih jauh penggunaan natbib, silahkan rujuk ke alamat : http://merkel.texture.rocks/Latex/natbib.php
%% \usepackage[colorlinks=true,linkcolor=blue,urlcolor=blue,citecolor=blue]{hyperref} : membuat hyperlink untuk referensi


%% Keterangan \newcommand :
%% ------------------------
%% \renewcommand{\chaptername}{BAB} : untuk mengubah "<Chapter Name>" menjadi "BAB"
%% \renewcommand{\bibname}{Daftar Pustaka} : untuk mengubah "<Bibliography Name>" menjadi "Daftar Pustaka"

%% Keterangan mode sitasi yang disediakan natbib :
%% -----------------------------------------------
%% \citet : textual citation
%% \citep : parenthetical citation
%% \citealt : citet tanpa kurung
%% \citealp : citep tanpa kurung
%% \citet* : \citet* versi nama  > 2 ditulis semua, bukan et.al
%% \citep* : \citep* versi nama  > 2 ditulis semua, bukan et.al
%% \citealt* : \citealt* versi nama > 2 ditulis semua, bukan et.al
%% \citealp* : \citealt* versi nama  > 2 ditulis semua, bukan et.al


%% Jenis-jenis style bibliography yang dapat digunakan :
%% -----------------------------------------------------
%% plain, abbrv, acm, alpha, apalike, ieeetr, siam, unsrt

%% Style bibliography yang kompatibel dengan natbib : 
%% --------------------------------------------------
%% plainnat, abbrvnat, unsrtnat