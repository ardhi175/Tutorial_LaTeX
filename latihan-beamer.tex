\documentclass{beamer}

\usepackage[bahasa]{babel}
\usetheme{CambridgeUS}

\author{M. Ardhi K}
\title{Pengenalan Beamer}
\date{\today}

\newtheorem{teoremas}{Teorema Saya}

\begin{document}
\maketitle

\section{Pendahuluan}
\begin{frame}
\frametitle{Judul frame ke 1}

Ini contoh tampilan enumerate di beamer :
\begin{enumerate}
\item Fisika \pause
\item Kimia \pause
\item Matematika \pause
\end{enumerate}

\end{frame}

\begin{frame}
%\frametitle{Judul frame ke 2}
Ini contoh tampilan itemize di beamer :
\begin{itemize}
\item Fisika \pause
\item Kimia
\item Matematika
\end{itemize}
\end{frame}

\section{Rumusan Masalah}
\begin{frame}
\frametitle{Judul frame ke 3}
Ini contoh tampilan itemize di beamer :
\begin{itemize}
\item Fisika
\item Kimia
\item Matematika
\end{itemize}
\end{frame}

\section{Pembahasan}
\begin{frame}
\frametitle{Judul frame teorema}
\begin{teoremas}
Let $f$ be a function whose derivative exists in every point, then $f$ 
is a continuous function.
\end{teoremas}
\end{frame}


\end{document}

%% Jenis-jenis theme beamer :
%% AnnArbor, Antibes, Bergen, Berkeley, Berlin, Boadilla, boxes, CambridgeUS, Copenhagen, Darmstadt, default, Dresden, Frankfurt, Goettingen, Hannover, Ilmenau, JuanLesPins, Luebeck, Madrid, Malmoe, Marburg, Montpellier, PaloAlto, Pittsburgh, Singapore, Szeged, Warsaw
%% Untuk tampilannya bisa dilihat di : https://deic-web.uab.cat/~iblanes/beamer_gallery/index_by_theme.html

%% Membuat tabel secara mudah https://tableconvert.com/